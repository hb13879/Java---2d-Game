\documentclass{article}

\begin{document}

\title{Making a 2d game}
\maketitle

section{Introduction}

section{Early stages}

Set up grid. Use private final SIZE and SPRITE. Add claim function and print function. Decide to make size of grid
in constructor since this will allow for easier (smaller scale) testing of certain functions (eg print, init).
The refactored the print method to a toString method ,overriding the standard method of the object class. This
can then be printed or autotested. Next add move function. Refactored cur_position from an int[] to a row and
column int for simplicity. Added test grid as a field. Add tests for move.

Consider how to display grid with graphics. Use group, make it variable size. Need to test? Made it fit the screen
it is being displayed on. Fiddled around a lot to parameterise all calcs.

Implemented move function to graphics end - move called from Cross.

Added simple coins. Had to add testmode parameter to grid since random initialisation of coins causes problems with
unit tests of grid based on strig output. Set so coin in testmode always appears in 0,0.

Added in score using text.

How to deal with sprites. Should it be a separate sprite class which stores the images and the position? But then
would have to be sent its position then just send it back.

refactored so each sprite is separate class. Cross is calling from grid and passing info to sprites. Then cross draws
game state.

Added enemies. Added timer.
